\begin{abstract}
The rapid shift to hybrid work environments has rendered traditional perimeter-based security models obsolete. As organizations distribute data and applications across cloud and on-premise infrastructures, the "castle-and-moat" approach fails to protect against sophisticated threats such as lateral movement and insider attacks. This research proposes and evaluates an AI-powered Zero Trust Architecture (ZTA) framework designed to enhance malware prevention and security posture in hybrid environments. By integrating continuous identity verification, micro-segmentation, and machine learning-based anomaly detection (Isolation Forest), the proposed framework enforces the principle of "never trust, always verify."

A comprehensive simulation was conducted to compare the proposed ZTA framework against a traditional baseline security model. The simulation modeled a realistic hybrid work environment with 50 users and 75 devices, subjecting both scenarios to various breach attempts including credential theft, lateral movement, and device compromise. The results demonstrate that the AI-powered ZTA framework achieved a 44.0\% improvement in breach prevention rate compared to the baseline, while maintaining a high level of user experience. These findings validate the effectiveness of Zero Trust principles augmented by AI in mitigating modern cybersecurity threats.
\end{abstract}