%% conclusion.tex
%%

%% ==================
\section{Conclusion and Future Work}
\label{sec:Conclusion}
%% ==================

This research set out to answer the question: \textit{"How effective is an AI-powered Zero Trust Architecture in preventing malware and lateral movement compared to traditional perimeter-based security?"} Through the design, implementation, and simulation of a comprehensive ZTA framework, this study has provided empirical evidence to answer this question.

The results demonstrate that the proposed AI-powered ZTA framework achieves a 44.0\% improvement in breach prevention compared to the baseline. By integrating continuous identity verification, strict device posture checks, and ML-based anomaly detection, the framework successfully mitigated high-risk threats such as lateral movement and credential theft that bypassed traditional defenses. Importantly, this security gain was achieved while maintaining a viable user experience.

\subsection{Future Work}
While this simulation provides strong evidence for ZTA, several avenues for future research remain:
\begin{itemize}
    \item \textbf{Real-world Deployment:} Implementing the framework in a live testbed environment to evaluate its performance against real-world network traffic and latency constraints.
    \item \textbf{Federated Learning:} Exploring the use of Federated Learning to train the AI models across distributed devices without centralizing sensitive user data, thus enhancing privacy.
    \item \textbf{Adversarial AI:} Investigating the resilience of the Isolation Forest model against adversarial attacks designed to poison the training data or evade detection.
\end{itemize}
