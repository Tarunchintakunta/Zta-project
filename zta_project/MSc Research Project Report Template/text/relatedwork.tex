%% content.tex
%%

%% ==============
\section{Related Work}
\label{sec:rw}
%% ==============

This section reviews the evolution of network security models, focusing on the transition from perimeter-based defenses to Zero Trust Architecture (ZTA), and the integration of Artificial Intelligence (AI) for enhanced threat detection.

\subsection{The Fall of Perimeter Security}
Traditionally, organizations relied on the "castle-and-moat" security model, which focused on defending the network perimeter. This model assumes that all internal traffic is trusted. However, the rise of cloud computing and remote work has dissolved the traditional perimeter. \cite{Kindervag2010} introduced the concept of Zero Trust to address these limitations, arguing that trust is a vulnerability and must be eliminated from network architecture.

\subsection{Zero Trust Architecture (ZTA)}
The National Institute of Standards and Technology (NIST) formalized ZTA in Special Publication 800-207 \cite{NIST2020}. NIST defines ZTA as a cybersecurity paradigm that focuses on resource protection and the premise that trust is never granted implicitly but must be continuously evaluated. Key tenets include:
\begin{itemize}
    \item All data sources and computing services are considered resources.
    \item Access to individual enterprise resources is granted on a per-session basis.
    \item Access to resources is determined by dynamic policy—including the observable state of client identity, application/service, and the requesting asset—and may include other behavioral and environmental attributes.
\end{itemize}
\cite{Buck2021} conducted a multivocal literature review on ZTA, highlighting that while the theoretical foundations are strong, there is a lack of empirical studies demonstrating its effectiveness in specific scenarios like hybrid work.

\subsection{AI in Cybersecurity}
Artificial Intelligence, particularly Machine Learning (ML), plays a crucial role in modern cybersecurity. Traditional rule-based systems struggle to detect novel attacks (zero-day exploits) or subtle insider threats. \cite{Sarker2021} provides a comprehensive overview of deep learning techniques in cybersecurity, noting their effectiveness in intrusion detection and malware analysis.

\subsection{Anomaly Detection with Isolation Forest}
For behavioral analysis, unsupervised learning algorithms are particularly valuable as they do not require labeled datasets of attacks. The Isolation Forest algorithm, proposed by \cite{Liu2018}, is an effective method for anomaly detection. Unlike distance-based methods, Isolation Forest explicitly isolates anomalies rather than profiling normal data points. This makes it highly efficient for detecting outliers in high-dimensional data, such as user behavior logs (login times, locations, resource access patterns).

\subsection{Summary}
While existing literature extensively covers ZTA principles and AI techniques separately, there is limited research that combines them into a cohesive framework and quantitatively evaluates their performance against traditional models in a simulated hybrid environment. This research aims to bridge this gap.