%% introduction.tex
%%

%% ==============================
\section{Introduction}
\label{sec:Introduction}
%% ==============================

The modern cybersecurity landscape has undergone a paradigm shift. The traditional network perimeter, once the cornerstone of enterprise security, has dissolved with the advent of cloud computing, mobile devices, and the widespread adoption of hybrid work models. Organizations no longer have a single "inside" to protect; instead, users, devices, and applications are distributed across the globe, accessing resources from untrusted networks. In this environment, the traditional "castle-and-moat" security model, which assumes implicit trust for anyone inside the network, is fundamentally flawed. Once an attacker breaches the perimeter, they are often free to move laterally, escalating privileges and exfiltrating sensitive data without detection.

The rapid evolution of cyber threats has rendered traditional perimeter-based security models obsolete. As organizations migrate to hybrid cloud environments and embrace remote work, the assumption of trust within the network boundary is no longer tenable. Zero Trust Architecture (ZTA) has emerged as the paradigm shift required to address these modern challenges, operating on the principle of "never trust, always verify."

A critical inspiration for this project comes from recent research by the Idaho National Laboratory (INL). In their paper titled \textit{"Developing an AI-Powered Zero-Trust Cybersecurity Framework for Malware Prevention in Nuclear Power Plants"} \citep{Talukder2023}, the authors highlight that in highly critical environments, traditional perimeter-based security is insufficient because even a single compromised device or insider can cause catastrophic damage. They propose combining AI algorithms with Zero Trust principles, meaning the system constantly verifies every request, user, and device, instead of trusting anything by default.

The framework described in the INL paper uses machine learning models to continuously monitor network traffic and user behavior. Instead of depending only on static access control lists or predefined roles, their AI model analyzes real-time behavior patterns to detect anomalies, such as sudden changes in file access, unexpected network commands, or unauthorized data transfers. Whenever the AI detects something suspicious, it automatically updates access rules or isolates the device. Essentially, the AI becomes a dynamic decision-maker.

\subsection{Project Connection: AI + Zero Trust}
The key idea we derive from this research is that AI enhances Zero Trust by adding adaptive, predictive security. While Zero Trust says "Never trust, always verify," AI adds "and keep learning while verifying." This ensures access decisions are data-driven, learning what is normal and what is risky even for authorized users.

In this project, the \textbf{Zero Trust Simulator}, we have adapted this concept to make access decisions intelligent. We have implemented modules for authentication, device posture checking, and segmentation. Building on the INL research, we added an AI-driven risk scoring model that analyzes the logs generated by our simulator. Every login, access attempt, device status, and network event is recorded in real-time using our \texttt{central_logger.py} module. We use this data as input for a machine learning model to learn behavioral patterns (e.g., login frequency, access times). Based on this, the AI assigns a risk score for each event:
\begin{itemize}
    \item \textbf{Low Score}: Allow access.
    \item \textbf{Medium Score}: Verify with Multi-Factor Authentication (MFA).
    \item \textbf{High Score}: Deny access or isolate the device.
\end{itemize}

This report details the design, implementation, and evaluation of this AI-enhanced Zero Trust framework...
Zero Trust Architecture (ZTA) has emerged as the definitive solution to these challenges. Grounded in the principle of "never trust, always verify," ZTA assumes that no entity---whether inside or outside the network---should be trusted by default. Every access request must be fully authenticated, authorized, and encrypted before granting access. Furthermore, the integration of Artificial Intelligence (AI) and Machine Learning (ML) into ZTA enables real-time threat detection and adaptive policy enforcement, allowing security systems to respond to anomalies at machine speed.

This research focuses on developing and evaluating an AI-powered ZTA framework specifically designed for malware prevention in hybrid work environments. By leveraging behavioral analytics and continuous monitoring, the framework aims to detect and block threats that bypass traditional defenses.

\subsection{Research Question}
The primary research question addressed in this study is:
\textit{"How effective is an AI-powered Zero Trust Architecture in preventing malware and lateral movement compared to traditional perimeter-based security in a hybrid work environment?"}

\subsection{Research Objectives}
To answer this question, the following objectives have been established:
\begin{itemize}
    \item \textbf{Objective 1:} To design a Zero Trust Architecture framework that integrates Identity and Access Management (IAM), micro-segmentation, and device trust verification.
    \item \textbf{Objective 2:} To implement an AI-driven anomaly detection engine using the Isolation Forest algorithm to identify malicious behavior patterns.
    \item \textbf{Objective 3:} To develop a simulation environment that models a realistic hybrid workforce, including users, devices, and applications.
    \item \textbf{Objective 4:} To conduct a comparative analysis between the proposed ZTA framework and a traditional baseline security model, measuring metrics such as breach prevention rate, security score, and usability impact.
\end{itemize}

\subsection{Contribution}
This research contributes to the field of cybersecurity by providing a quantifiable evaluation of ZTA effectiveness. Unlike theoretical models, this study implements a functional simulation to generate empirical data on breach prevention. It demonstrates how AI can enhance ZTA by reducing false positives and detecting novel attacks that rule-based systems might miss.

\subsection{Report Structure}
The remainder of this report is organized as follows: Section~\ref{sec:rw} reviews related work in Zero Trust and AI-driven security. Section~\ref{sec:Methodology} details the research methodology and simulation setup. Section~\ref{sec:Design} presents the design specification of the proposed framework. Section~\ref{sec:Implementation} describes the implementation details. Section~\ref{sec:Evaluation} presents the experimental results and analysis. Finally, Section~\ref{sec:Conclusion} concludes the report and outlines future work.
